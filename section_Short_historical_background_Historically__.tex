\section{Short historical background}

Historically, a major source of data to derive incidence estimates were results from tuberculin surveys conducted in children\cite{Styblo1985}. Early studies showed  the following relationship between the annual risk of infection denoted $\lambda$ and the incidence of smear positive TB denoted $I^+$: one smear positive case infects on average 10 individuals per year for a period of 2 years and an annual risk of infection of 1\% corresponds approximately to an incidence of 50 smear positive cases per 100 000 per year.
\begin{equation}
\label{eqn:ari}
I^+ = \frac{\lambda \times 10^5}{2 \times 10}
\end{equation}

However, the above relationship no longer holds in the context of modern TB control and in HIV settings \cite{18235886}. In addition to uncertainty about the relationship between $\lambda$ and $I^+$, estimates of incidence obtained from tuberculin surveys suffer from other sources of uncertainty, including unpredictable diagnostic performance of the tuberculin test, digit preference when reading and recording the size of tuberculin reactions, sensitivity to assumptions about reaction sizes attributed to infection, sensitivity to the common assumption that the annual risk of infection is age invariant, and lastly, sensitivity of overall TB incidence estimates to the assumed proportion of TB incidence that is smear positive. 

A first global and systematic estimation exercise led by WHO in the early 1990s estimated that there were approximately 8 million incident TB cases in 1990 ($152 \times 10^{-5} y^{-1}$) and 2.6-2.9 million deaths ($46-55 \times 10^{-5} y^{-1}$) \cite{1600578}. A second major reassessment was published in 1999 \cite{10517722}, with an estimated 8 million incident  cases for the year 1997 ($136 \times 10^{-5} y^{-1}$), and 1.9 million TB deaths ($32 \times 10^{-5} y^{-1}$). The most important sources of information were case notification data for which gaps in detection and reporting were obtained from expert opinion. In addition, data from 24 tuberculin surveys were translated into incidence using equation (\ref{eqn:ari}) and 14 prevalence surveys of TB disease were used.

Starting in 1997, global TB reports were published annually by WHO, providing updated data on case notifications and estimated TB burden. In June 2006, the WHO Task Force on TB Impact Measurement was established \cite{18201929} (see Chapter 2, Box 2.1). The Task Force subgroup on TB estimates reviewed methods and provided recommendations in 2008, 2009 and most recently in March 2015. Methods described in this document reflect short-term recommendations from the 2015 review. 




  
  
  
  
  
  
  
  
  
  
  