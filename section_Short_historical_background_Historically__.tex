\section{Short historical background}

Historically, a major source of data to derive incidence estimates consisted in results from tuberculin surveys conducted in children\cite{Styblo1985}, showing the following relationship between the annual risk of infection denoted $\lambda$ and the incidence of smear positive TB denoted $I^+$: one smear positive case infects on average 10 individuals per year for a period of 2 years.

$I^+ = \frac{\lambda \times 10^5}{2 \times 10} = 50.10^{-5} y^{-1}$

However, the above relationship no longer holds in the context of modern TB control or in HIV settings \cite{18235886}. Estimates of incidence obtained from tuberculin surveys suffer from multiple sources of uncertainty in addition to the above uncertain relationship, including poor diagnostic performance of the tuberculin test, digit preference where reading the size of tuberculin reactions, sensitivity to assumptions about the distribution of reaction sizes attributed to false positive tuberculin results, sensitivity to the assumption of age invariance of the risk of infection and the proportion of overall TB incidence that is smear positive. 

A first global and systematic estimation exercise led by WHO estimated that there were 8 million incident TB cases in 1990 (152 per 100,000) and 2.6-2.9 million deaths (46-55 per 100,000) \cite{1600578}. A second major reassessment published nine years later \cite{10517722} estimated an incident 8 million cases for the year 1997 (136 per 100,000), with 1.9 million TB deaths (32 per 100,000). By and large, the most important source of information consisted in case notification data for which gaps in detection and reporting were estimated through elicitation of expert opinion. In addition, data from 24 tuberculin surveys and 14 prevalence surveys of TB disease were used.

Starting in 1997, global TB reports where published annually by WHO, providing updated data on TB burden. In June 2006, the WHO Task Force on TB Impact Measurement was established \cite{18201929}. The subgroup on TB estimates met in three times to evaluate methods and provide recommendations, in 2008, 2009 and most recently in March 2015 (see Chapter 2, Box 2.1).

The best sources of data on TB burden are 
\begin{itemize}
\item TB notifications when data meet quality criteria and under-reporting low and documented;
\item TB mortality from Vital Registration with Cause of death data coded according to standard rules of the International Classification of Diseases ($10^{th}$ Edition);
\item Prevalence from population-based national prevalence surveys.
\end{itemize}




  
  
  
  
  
  
  
  
  
  
  