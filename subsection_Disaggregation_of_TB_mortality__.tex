\subsection{Disaggregation of TB mortality by age and sex}

\subsubsectiom{TB deaths among HIV-negative people}

From the age-specific adjusted (for coverage and ill-defined causes) number of deaths from VR, we first estimated the ratio $r_2$ of rates in children ($M_{0-14}$) compared to adults ($M_{15+}$) (\ref{eqn:r2}). The overall mortality rate for all ages ($M$) can be expressed as a weighted average of mortality in children and adults, where  is the proportion of children among the general population (\ref{eqn:M}).

\begin{equation}
\begin{align}
r_2 &= \frac{M_{0-14}}{M_{15+}} \label{eqn:r2} \\
M &= c M_{0-14} + (1 - c) M_{15+} \label{eqn:M}
\end{align}
\end{equation}

$M_{0-14}$ for countries with VR data is directly measured. For countries without VR data an imputation step is necessary where the ratio $r$ is predicted from a regression model with risk factors known to be associated with TB. For the sex disaggregation of TB mortality among adults ($M_{15+}$), we use sex-specific adjusted (for coverage and ill-defined causes) number of deaths from VR to estimate mortality rates in men $M_m$  and women $M_w$ (\ref{eqn:mm}). The ratio of these rates $r_3$ (\ref{eqn:r3}) is either directly measured in countries with VR data or imputed in countries without.

\begin{equation}
\begin{align}
M_{15+} &= M_w + M_m \label{eqn:mm} \\
r_3 &= \frac{M_m}{M_w} \label{eqn:r3}
\end{align}
\end{equation}


