\section{Historical background}

The best sources of data on TB burden are 
\begin{itemize}
\item TB notifications when data meet quality criteria and under-reporting low and documented
\item TB mortality from Vital Registration with Cause of death data coded according to standard rules of the International Classification of Diseases ($10^{th}$ edition).
\item Prevalence from population-based national prevalence surveys
\end{itemize}

Historically, incidence has been estimated using results from tuberculin survey results\cite{Styblo1985}, showing the following relationship between the annual risk of infection denoted $\lambda$ and the incidence of smear positive TB denoted $I$.

$I^+ = \frac{\lambda \times 10^5}{2 \times 10} = 50.10^{-5} y^{-1}$

However, the above relationship no longer holds in the context of modern TB control or in HIV settings \cite{18235886}. Estimates of incidence obtained from tuberculin surveys suffer from multiple sources of uncertainty in addition to the above uncertain relationship, including poor diagnostic performance of the tuberculin test, digit preference where reading the size of tuberculin reactions, sensitivity to assumptions about the distribution of reaction sizes attributed to false positive tuberculin results, sensitivity to the assumption of age invariance of the risk of infection and the proportion of overall TB incidence that is smear positive. 

First WHO systematic estimation exercise\cite{1600578}.

Global reassessment\cite{10517722}.

WHO Task Force on TB Impact Measurement\cite{18201929}.



  
  
  
  
  
  
  
  
  
  
  