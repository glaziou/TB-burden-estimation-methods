\subsection{Estimating TB mortality among HIV-negative people from vital registration data and mortality surveys}

Data from VR systems are reported to WHO by Member States and territories every year. In countries with functioning VR systems in which causes of death are coded according to the two latest revisions of the International classification of diseases (underlying cause of death: ICD-10 A15-A19, equivalent to ICD-9: 010-018), VR data are the best source of information about deaths from TB among people not infected with HIV. When people with AIDS die from TB, HIV is registered as the underlying cause of death and TB is recorded as a contributory cause. Since one third of countries with VR systems report to WHO only the underlying causes of death and not contributory causes, VR data usually cannot be used to estimate the number of TB deaths in HIV-positive people. 

TB mortality data obtained from VR systems are essential to understand trends in TB disease burden where case notifications have incomplete coverage or their coverage is not documented through an inventory study. 

As of July 2015, 130 countries had reported mortality data to WHO (including data from sample VR systems and mortality surveys). These 130 countries included 10 of the 22 high TB burden countries (HBCs): Brazil, China, India, Indonesia, the Philippines, the Russian Federation, South Africa, Thailand, Viet Nam and Zimbabwe. However, the VR data on TB deaths from Zimbabwe were not used for this report because large numbers of HIV deaths were miscoded as TB deaths. Improved empirical adjustment procedures have recently been published\cite{21479092}. Estimates for South Africa adjusted for HIV/TB miscoding were obtained from the Institute of Health Metrics and Evaluation at \url{http://vizhub.healthdata.org/cod/}. 

Among the countries for which VR data could be used (see Figure \ref{fig:vr}), there were 2361 country-year data points 1990–2014, after 13 outlier data points from systems with very low coverage ($<20\%$) or very high proportion of ill-defined causes ($>50\%$) were excluded for analytical purposes. The median number of data points per country was 21 (IRQ 15 - 23). 

Reports of TB mortality were adjusted upwards to account for incomplete coverage (estimated deaths with no cause documented) and ill-defined causes of death (ICD-9 code B46, ICD-10 codes R00–R99).\cite{15798840}

It was assumed that the proportion of TB deaths among deaths not recorded by the VR system was the same as the proportion of TB deaths in VR-recorded deaths. For VR-recorded deaths with ill-defined causes, it was assumed that the proportion of deaths attributable to TB was the same as the observed proportion in recorded deaths. 

The adjusted number of TB deaths $d_a$ was obtained from the VR report $d$ as follows:

\begin{align*}
d_a = \frac{d}{c(1-g)}
\end{align*}

where $c$ denotes coverage (i.e. the number of deaths with a documented cause divided by the total number of estimated deaths) and $g$ denotes the proportion of ill-defined causes.

The uncertainty related to the adjustment was estimated with standard deviation $\textrm{SD} = d/4 [1/c(1-g) -1]$. The uncertainty calculation does not account for miscoding, such as HIV deaths miscoded as deaths due to TB. 

Missing data between existing adjusted data points were interpolated. Trailing missing values were predicted using exponential smoothing models for time series. A penalized likelihood method based on the in-sample fit was used for country-specific model selection. Leading missing values were similarly predicted backwards to 1990. A total of 1076 country-year data points were thus imputed.

Results from mortality surveys were used to estimate TB mortality in India and Viet Nam. 

In 2014, 43\% of global TB mortality (excluding HIV) was directly measured from VR or survey data (or imputed from survey or VR data from previous years). The remaining 57\% was estimated using the indirect methods described in the next section.




