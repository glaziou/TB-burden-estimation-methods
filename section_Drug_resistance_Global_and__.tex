\section{Drug resistance}

Global and regional estimates of the proportion of new and retreatment cases of TB that had MDR-TB in 2014 were calculated using country-level information. If countries had reported data on the proportion of new and retreatment cases of TB that have MDR-TB from routine surveillance or a survey of drug resistance the latest available information was used. For data from routine surveillance to be considered representative, at least 60\% of notified new pulmonary TB cases must have a documented DST result for at least rifampicin. For countries that have not reported such data, estimates of the proportion of new and retreatment cases of TB that have MDR-TB were produced using modelling (including multiple imputation) that was based on data from countries for which data do exist. Estimates for countries without data were based on countries that were considered to be similar in terms of TB epidemiology (for country groups see Appendix 1). The observed and imputed estimates of the proportion of new and retreatment cases of TB that have MDR-TB were then pooled to give a global estimate, with countries weighted according to their share of global notifications of new and retreatment cases.

\subsection{MDR-TB mortality}

The VR mortality data reported to WHO by Member States does not differentiate between MDR-TB and non-MDR-TB as a cause of death (there is no specific ICD-9 or ICD-10 codes for MDR-TB, although countries such as South Africa have allocated two specific codes \textit{U51} and \textit{U52} to classify deaths from MDR-TB and XDR-TB respectively). Therefore, a systematic review and meta-analysis of the published literature was undertaken to estimate the relative risk of dying from MDR-TB compared with non MDR-TB. The global estimate of MDR-TB deaths (Chapter 2) was then based on the following formula: 

$m = M.p.r$ 

Where:
$m$ = global MDR-TB mortality,
$M$ = global TB mortality,
$p$ = overall proportion of MDR-TB among prevalent TB cases, approximated by the weighted average of the proportion of new and retreated cases that have MDR-TB,
$r$ = the relative risk of dying from MDR-TB versus non-MDR-TB.

\subsection{Number of incident cases of MDR-TB}

The global estimate of MDR-TB incidence was calculated as the addition of three groups of MDR-TB incident cases:
\begin{enumerate}
\item incident MDR-TB among new pulmonary and extra-pulmonary incident TB cases, using the proportion of MDR-TB among new cases from drug resistance surveillance (DRS); 
\item incident MDR-TB among relapses, using the proportion of MDR-TB among new cases from DRS and the estimated relative risk of MDR among relapse versus new cases; and
\item incident MDR-TB among retreated cases that are not relapses, which was assumed to follow a uniform distribution with min=0, max=upper limit of the global proportion of MDR-TB among retreated cases estimated from DRS. 
\end{enumerate}


\subsection{Resistance to second-line drugs among patients with MDR-TB}

Data from 75 countries were used to produce global estimates of the following proportions: (i) patients with MDR-TB who had XDR-TB; (ii) patients with MDR-TB who had fluoroquinolone resistance; (iii) patients with MDR-TB who had resistance to second-line injectable drugs and fluoroquinolones but not XDR-TB. The latest available national and subnational data from each country were analysed using logistic regression models with robust standard errors to account for the clustering effect at the level of the country or territory. The analysis was limited to countries in which more than 66\% of MDR-TB cases received second-line DST.




