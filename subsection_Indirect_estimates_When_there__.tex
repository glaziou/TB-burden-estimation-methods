\subsection{Indirect estimates}
When there is no direct measurement from a national survey of the prevalence of TB disease, prevalence is the most uncertain of the three TB indicators used to measure disease burden. This is because prevalence is the product of two uncertain quantities: (i) incidence and (ii) disease duration. The duration of disease is very difficult to quantify because it cannot be measured during surveys of the prevalence of TB disease (surveys truncate the natural history of disease). Duration can be assessed in self-presenting patients, but there is no practical way to measure the duration of disease in patients who are not notified to NTPs.

Indirect estimates of prevalence were calculated according to the following equation:

$P = \sum I_{i,j} d_{i,j}, i \in \lbrace 1, 2\rbrace, j \in \lbrace 1, 2\rbrace$

where the index variable $i$ denotes HIV+ and HIV–, the index variable $j$ denotes notified and non-notified cases, $d$ denotes the duration of disease in notified cases and $I$ is total incidence. In the absence of measurements, we did not allow duration in notified cases to vary among countries. Given their underlying uncertainty, prevalence estimates should be used with great caution in the absence of direct measurements from a prevalence survey. Unless measurements were available from national programmes (for example, Turkey), assumptions of the duration of disease were used as shown in Table \ref{tab:duration}.

Scarce empirical data on disease duration (of note, a typically large proportion of bacteriologically confirmed cases detected during TB prevalence surveys did not report symptoms suggestive of TB at the time of survey investigations). An important limitation is that duration is considered constant within case categories for all settings and over time.