The above two methods to derive incidence from prevalence are compared below in Table \ref{tab:2methods}.

\begin{table} 
    \begin{tabular}{ c c c c }
    \hline
         & Prevalence & Incidence - method 1 & Incidence - method 2 \\ 
         & $(10^{-3})$  & $(10^{-3} y^{-1})$     & $(10^{-3} y^{-1})$ \\
    \hline
        Cambodia 2002 & 12 (10-15) & 4 (2.5-5.8) & 2.2 (1.5 – 2.9) \\ 
        Cambodia 2011 & 8.3 (7.1-9.8) & 6.7 (4.5-9.3) & 3.8 (2.2 – 5.8) \\ 
        Myanmar 2009 & 6.1 (5-7.5) & 3.3 (2-4.8) & 3.4 (2 – 5.1) \\ 
        Thailand 2012 & 2.5 (1.9-3.5) & 2.3 (1-3.5) & 1.1 (0.7 – 1.6) \\ 
    \hline
    \end{tabular} 
    \caption{Estimates of incidence derived from prevalence survey results, based on two estimation methods.} 
    \label{tab:2methods}
\end{table}

It is not clear which method will perform better, validation would require a measurement of incidence. The second method requires a sufficient number of cases on treatment at the time of the survey (as a rule of thumb, at least 30 cases) to generate stable estimates. When the number of cases on treatment is too small, the amount of propagated uncertainty renders estimates of incidence unusable and only the first approach is used. 

If both methods can be applied, results from two methods may be combined in a statistical ensemble approach as follows:

The incidence rate obtained using method $i$ is assumed distributed Beta with shape and scale parameters $\alpha_i + 1$ and $\beta_i + 1$, respectively, and determined using the method of moments based on equation \ref{eqn:betamoments}: $I_i \sim B(\alpha_i + 1, \beta_i + 1)$ so that 

\begin{align*}
Prob(x = \textrm{TB})= \int_{0}^{1} x B(\alpha_i, \beta_i) dx = \frac{\alpha_i+1}{a_i+\beta_i+2}
\end{align*}

The combined probability is then expressed as 

\begin{align*}
Prob(x = \textrm{TB}) = \frac{\sum{\alpha_i}+1}{\sum{\alpha_i}+\sum{\beta_i}+2} 
\end{align*}

