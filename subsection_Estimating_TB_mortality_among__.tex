\subsection{Estimating TB mortality among HIV-negative people from estimates of case-fatality rates and TB incidence
}

In 88 countries lacking VR data of the necessary coverage and quality, TB mortality was estimated as the product of TB incidence and the CFR after disaggregation by case type as shown in Table \ref{tab:cfr}, following a literature review of CFRs by the TB Modelling and Analysis Consortium (TB-MAC):

$M = (I-N)F_u + NF_t$ , where $M$ denotes mortality, $I$ incidence. $F_u$ and $F_t$ denote CFRs untreated and treated, respectively. $N$ denotes the number of notified TB cases. In countries where the number of treated patients that are not notified (under-reporting) is known from an inventory study, $N$ is adjusted upwards to account for under-reporting. 

\begin{table} 
    \begin{tabular}{ c c c }
    \hline
         & CFR & Sources \\ 
         \hline
        Not on TB treatment $F_u$ & 0.43 (0.28- 0.53) &  \cite{12742798} \cite{21483732} \\ 
        On TB treatment $F_t$ & 0.03 (0.00-0.07) &  \cite{21738585} \\ 
        \hline
    \end{tabular} 
    \caption{Distribution of CFRs by case category} 
    \label{tab:cfr}
\end{table}

This approach tends to overestimate TB mortality in countries with no VR or mortality survey data and the level of under-reporting of treated TB cases is unknown and large relative to the number of reported cases. 

Figure \ref{fig:mvalidity} shows a comparison of 129 VR-based mortality estimates for 2014 and indirect estimates obtained from the CFR approach for the same countries. It should be noted that countries with VR data tend to be of a higher socio-economic status compared with countries with no VR data where the indirect approach was used.


