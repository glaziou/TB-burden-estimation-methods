\subsection {Notifications in high-income countries adjusted by a standard factor to account for under-reporting and under-diagnosis.} This method is used for 73 countries (all high-income countries except the Netherlands and the United Kingdom), which accounted for 3\% of the estimated global number of incident cases in 2014.

In the absence of country-specific data on the quality and coverage of TB surveillance systems, it was assumed that TB surveillance systems from countries in the high-income group performed similarly well, although the model does allow for stochastic fluctuations. The exceptions were the United Kingdom and the Netherlands, where the underreporting of TB cases has been measured using inventory studies and capture–recapture modelling\cite{Anderson2010}\cite{17156496}. For these two countries, the results from these studies were used to measure TB incidence directly.

Surveillance data in this group are usually internally consistent. Consistency checks include detection of rapid fluctuations in the ratio of TB deaths / TB notifications ($M/N$ ratio), which may be indicative of reporting problems, accounting for stochastic fluctuations.



\subsection {Results from inventory/capture-recapture studies\cite{WHO2012}.} This method is used for 5 countries: Egypt, Iraq, the Netherlands, the United Kingdom and Yemen. They accounted for 0.5\% of the estimated global number of incident cases in 2014. Capture-recapture (CR) modelling was considered in studies with at least 3 lists and estimation of list dependencies. The CR-based estimate of the surveillance gap in the UK and the Netherlands was assumed time invariant. In Yemen, trends were derived from results of repeat tuberculin surveys (for details about the approach, see group 5 below). 

