\section{Prevalence}

\subsection{Population-based surveys}
The best way to measure the prevalence of TB is through national population-based surveys of TB disease\cite{18713496}\cite{WHO2011}. Data from such surveys are available for an increasing number of countries and were used for 21 countries (Figure \ref{fig:ps}). It should be noted, however, that measurements of prevalence are typically confined to the adult population. Furthermore, prevalence surveys exclude extrapulmonary cases and do not allow the diagnosis of cases of culture-negative pulmonary TB. 

TB prevalence all forms and all ages ($P$) is measured as: pulmonary bacteriologically-confirmed TB prevalence ($P_{pulm}$) among those aged ≥15 measured from national survey ($P_{adult}$), adjusted for pulmonary TB in children ($P_{child}$) and extra-pulmonary TB all ages ($P_{ep}$):

\begin{align*}
P_{pulm} = c P_{child} + (1 − c) P_{adult}
\end{align*}

where $c$ is the proportion of children among the total country population.

\begin{align*}
P = \frac{P_{pulm}}{1 - P_{ep}}
\end{align*}

Surveys are logistically demanding, therefore suboptimal quality of prevalence survey data (e.g. low participation rate, missing lab results) may result in biases of estimates. Sampling uncertainty (relative precision is typically about 20\%), and most surveys are not powered to compute subpopulation estimates with precision. The  quality of routine surveillance data to inform levels of pulmonary TB in children and extra-pulmonary TB for all ages is often questionable.



 