\tableofcontents

\begin{abstract}
\addcontentsline{toc}{section}{Abstract}
This document complements Global TB Report 2015 and provides case definitions and methodological details used by WHO to estimate TB incidence, prevalence and mortality, age and sex disaggregations of indicators and indicators of MDR-TB burden. Incidence and mortality are disaggregated by HIV status. Four main methods are used to derive incidence: (i) case notification data combined with expert opinion about case detection gaps (120 countries representing 51\% of global incidence); (ii) results from national TB prevalence surveys (19 countries, 46\% of global incidence); (iii) notifications in high-income countries adjusted by a standard factor to account for under-reporting and under-diagnosis (73 countries, 3\% of global incidence) and (iv) capture recapture modelling (5 countries, 0.5\% of global incidence). Prevalence was obtained from results of national prevalence surveys in 21 countries, representing 69\% of global prevalence). In other countries, prevalence was estimated from incidence and disease duration. Mortality was obtained from national measurements (from vital registration systems of mortality surveys) in 129 countries (43\% of global HIV-negative TB mortality). In other countries, mortality is derived indirectly from incidence and the case fatality ratio.
\end{abstract}



\section{Introduction}

Estimates of the burden of disease caused by TB and measured in terms of incidence, prevalence and mortality are produced annually by WHO using information gathered through surveillance systems (case notifications and death registrations), special studies (including surveys of the prevalence of disease, mortality surveys, surveys of under-reporting of detected TB and in-depth analysis of surveillance data, expert opinion and consultations with countries. This document provides case definitions and describes the methods used in Global TB Report 2015 to derive TB incidence, prevalence and mortality.

\section{Definitions}
\textbf{Incidence} is defined as the number of new and recurrent (relapse) episodes of TB (all forms) occurring in a given year. Recurrent episodes are defined as a new episode of TB in people who have had TB in the past and for whom there was bacteriological confirmation of cure and/or documentation that treatment was completed. In the remainder of this technical document, relapse cases are referred to as recurrent cases because the term is more useful when explaining the estimation of TB incidence. Recurrent cases may be true relapses or a new episode of TB caused by reinfection. In current case definitions, both relapse cases and patients who require a change in treatment are called \textit{retreatment cases}. However, people with a continuing episode of TB that requires a treatment change are prevalent cases, not incident cases.

\textbf{Prevalence} is defined as the number of TB cases (all forms) at a given point in time. 

\textbf{Mortality} from TB is defined as the number of deaths caused by TB in HIV-negative people, according to the latest revision of the International classification of diseases (ICD-10). TB deaths among HIV-positive people are classified as HIV deaths in ICD-10. For this reason, estimates of deaths from TB in HIV-positive people are presented separately from those in HIV-negative people.

The \textbf{case fatality rate} is the risk of death from TB among people with active TB disease.

The \textbf{case notification} rate refers to new and recurrent episodes of TB notified to WHO for a given year, expressed per 100 000 population. The case notification rate for new and recurrent TB is important in the estimation of TB incidence. In some countries, however, information on treatment history may be missing for some cases. When data on treatment history are not available, recurrent cases cannot be distinguished from cases whose treatment was changed, since both are registered and reported in the category \textit{retreatment}. Patients reported in the \textit{unknown history} category are considered incident TB episodes (new or relapse). 

\textbf{Regional analyses} are generally undertaken for the six WHO regions (that is, the African Region, the Region of the Americas, the Eastern Mediterranean Region, the European Region, the South-East Asia Region and the Western Pacific Region). For analyses related to MDR-TB, nine epidemiological regions were defined (Figure \ref{fig:epiregions}). These were African countries with high HIV prevalence, African countries with low HIV prevalence, Central Europe, Eastern Europe, high-income countries, Latin America, the Eastern Mediterranean Region (excluding high-income countries), the South-East Asia Region (excluding high-income countries) and the Western Pacific Region (excluding high-income countries). 


  