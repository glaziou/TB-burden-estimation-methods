\subsubsection{Disaggregation by age and sex}

This section describes how estimates of TB incidence and TB mortality are disaggregated by age and sex. Specifically, estimates are estimated for men (defined as males aged $\geq 15$ years), women (defined as females aged $\geq 15$ years) and children (defined as people aged $< 15$ years). The cut-off of 15 years is used because it is consistent with the age categories for which notification data are reported and with the cut-off used in current guidelines to define people eligible to participate in a TB prevalence survey\cite{WHO2011}.

\subsubsubsection{TB incidence disaggregated by age} 
Age and sex disaggregation of acid-fast smear-positive tuberculosis case notifications has been requested from countries since the establishment of the data collection system in 1995, but with few countries actually reporting these data to WHO. In 2006, the data collection system was revised to additionally monitor age disaggregated notifications for smear-negative and extrapulmonary tuberculosis. The revision also included a further disaggregation of the 0–14 age group category to differentiate the very young (0–4) from the older children (5–14). While reporting of age disaggregated data was limited in the early years of the data collection system, coverage kept improving until for 2012 case notifications it reached 99\%, 83\% and 83\% out of total acid-fast smear-positive, smear-negative and extrapulmonary tuberculosis case notifications notified respectively that were age and sex disaggregated. Finally in 2013, another revision of the recording and reporting system was necessary to allow for the capture of cases diagnosed using WHO-approved rapid diagnostic tests (such as Xpert MTB/RIF). This current revision requests the reporting of all new and relapse case notifications by age and sex (but not separately by case type). The countries that reported age-disaggregated data in 2014 can be seen in Figure. 