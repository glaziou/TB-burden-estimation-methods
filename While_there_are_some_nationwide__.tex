While there are some nationwide surveys that have quantified the amount of under-reporting of cases diagnosed in the health sector outside the network of the NTPs\cite{20487611}\cite{17156496}hest\cite{18346285}, none have produced precise enough age-disaggregated results. Small-scale, convenient-sampled studies in some settings indicate that under-reporting of childhood tuberculosis is very high\cite{21985569}\cite{Coghlan2015-xn} but extrapolation to nationally representative, regional and global settings is not yet possible. This shortcoming is currently being addressed through the plans for implementation of national scale surveys in high priority countries in Asia to measure under-reporting of tuberculosis in children\cite{noauthor_2014-gv}.

Producing estimates of TB incidence among children is challenging primarily due to the lack of child-friendly tools to confirm diagnosis of TB and the lack of age-specific, nationwide, robust survey and surveillance data. However, progress is being made, based on collaborations established in 2013 between WHO and academic groups working on the estimation of TB disease burden among children, as well as recommendations from a global consultation on methods to estimate TB disease burden held earlier in 2015. As a result, methods to estimate TB incidence were updated for this report compared with those used to produce estimates published in 2013 and 2014. The updated methods involve use of a statistical ensemble approach in which results from two independent methods are combined with the original WHO approach that featured in the 2012 Global TB Report. 

The first method is based on the original WHO approach that estimated incidence of TB among children using case notification data among ages 0-14 combined with expert opinion about case detection gaps (as described in section 4.1.1) assuming these were the same in children as in TB cases of all ages. For the first time in this year’s report child specific case detection gaps are being used as estimated according to a previously published method\cite{Jenkins_2014} that has been updated to use more recent notification and other available data\cite{Sismanidis_2014}. This method estimates the proportion of all TB cases that are in children as a function of expected age-specific proportions of smear-positive TB among different age groups. 