\section{Incidence}

TB incidence has never been measured at national level because this would require long-term studies among large cohorts of people (hundreds of thousands), involving high costs and challenging logistics. Notifications of TB cases provide a good proxy indication of TB incidence in countries that have both high-performance surveillance systems (for example, there is little underreporting of diagnosed cases) and where the quality of and access to health care means that few cases are not diagnosed. In the large number of countries where these criteria are not yet met, better estimates of TB incidence can be obtained from an inventory study\cite{WHO2012} (an inventory study is a survey to quantify the level of underreporting of detected TB cases; if certain conditions are met, capture-recapture methods can also be used to estimate TB incidence). To date, such studies have been undertaken in only a few countries: examples include Egypt, Iraq, Pakistan and Yemen. 

The ultimate goal is to directly measure TB incidence from TB notifications in all countries. This requires a combination of strengthened surveillance, better quantification of underreporting (i.e. the number of cases that are missed by surveillance systems) and universal access to health care. A TB surveillance checklist developed by the WHO Global Task Force on TB Impact Measurement defines the standards that need to be met for notification data to provide a direct measure of TB incidence. By August 2015, a total of 38 countries including 16 HBCs had completed the checklist.

 Methods currently used by WHO to estimate TB incidence can be grouped into four major categories (Figure 2.2). These are: 

\begin{enumerate}
\item {Case notification data combined with expert opinion about case detection gaps.} Expert opinion, elicited in regional workshops or country missions, is used to estimate levels of under-reporting and under-diagnosis. Trends are estimated using either mortality data, surveys of the annual risk of infection or exponential interpolation using estimates of case detection gaps for three years. In this report, this method is used for 120 countries that accounted for 51\% of the estimated global number of incident cases in 2014. The estimation of case detection gaps is essentially based on an in-depth analysis of surveillance data, experts provide their educated best guess about the range of plausible detection gap $g$:

$I=\frac{f(N)}{1-g}, g\in[0,1[$

where $f$ denotes a cubic spline function in countries with large year-to-year fluctuations in $N$, or else, the identity function. The incidence series are completed using assumptions about improvements in CFR over time in countries with evidence of improvements in TB control performance such as an increased detection coverage over time or improved treatment outcomes:

$0 < \lvert \frac{dI}{dt} \rvert < \lvert \frac{dM}{dt} \rvert $

\item {Results from national TB prevalence surveys.} Incidence is estimated using prevalence survey results combined with either a dynamic model or estimates of the duration of disease. This method is used for 19 countries that accounted for 46\% of the estimated global number of incident cases in 2014.
\item {Notifications in high-income countries adjusted by a standard factor to account for under-reporting and under-diagnosis.} This method is used for 73 countries (all high-income countries except the Netherlands and the United Kingdom), which accounted for 3\% of the estimated global number of incident cases in 2014.
\item {Results from inventory/capture-recapture studies.} This method is used for 5 countries: Egypt, Iraq, the Netherlands, the United Kingdom and Yemen. They accounted for 0.5\% of the estimated global number of incident cases in 2014. 
\end{enumerate}




  
  