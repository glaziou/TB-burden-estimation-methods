\section{Prevalence}

The best way to measure the prevalence of TB is through national population-based surveys of TB disease\cite{18713496}\cite{WHO2011}. Data from such surveys are available for an increasing number of countries (Chapter 2). It should be noted, however, that measurements of prevalence are typically confined to the adult population. Furthermore, prevalence surveys exclude extrapulmonary cases and do not allow the diagnosis of cases of culture-negative pulmonary TB. 

When there is no direct measurement from a national survey of the prevalence of TB disease, prevalence is the most uncertain of the three TB indicators used to measure disease burden. This is because prevalence is the product of two uncertain quantities: (i) incidence and (ii) disease duration. The duration of disease is very difficult to quantify because it cannot be measured during surveys of the prevalence of TB disease (surveys truncate the natural history of disease). Duration can be assessed in self-presenting patients, but there is no practical way to measure the duration of disease in patients who are not notified to NTPs.

Indirect estimates of prevalence were calculated according to the following equation:

$P = \sum I_{i,j} d_{i,j}, i \in {1, 2}, j \in {1, 2}$


  