\subsubsection {Notifications in high-income countries adjusted by a standard factor to account for under-reporting and under-diagnosis.} This method is used for 73 countries (all high-income countries except the Netherlands and the United Kingdom), which accounted for 3\% of the estimated global number of incident cases in 2014.

In the absence of country-specific data on the quality and coverage of TB surveillance systems, it was assumed that TB surveillance systems from countries in the high-income group performed similarly well, although the model does allow for stochastic fluctuations. The exceptions were Korea (see Chapter 2, box 2.3), where the underreporting of TB cases has recently been measured using annual inventory studies and France, where the estimated level of under-reporting was communicated by public health authorities, based on unpublished survey results.

Surveillance data in this group are usually internally consistent. Consistency checks include detection of rapid fluctuations in the ratio of TB deaths / TB notifications ($M/N$ ratio), which may be indicative of reporting problems, accounting for stochastic fluctuations.



