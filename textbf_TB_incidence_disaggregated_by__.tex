\textbf{TB incidence disaggregated by sex}. Using the sex disaggregated reporting of TB case notification data we calculated the ratio of the number of TB cases notified in men compared with women as a measure of the ratio $r_1$ for incident cases, assuming no sex differential in the detection of incident cases. Evidence from national prevalence surveys of bacteriologically-positive pulmonary TB consistently show bigger recording and detection gaps in men as suggested by consistently higher prevalence to case notifications ratios in men compared with women\cite{Onozaki_2015}. This suggests that our assumption of no sex differential in the detection of incident cases may lead to underestimating the proportion of men among incident cases. With currently available data, it is not possible to estimate male and female case detection ratios for all countries.    

Overall incidence in adults 15 years or over ($I_a$) can be disaggregated into estimates among men ($I_m$) and women ($I_w$) as shown

\begin{align}
r_1 &= \frac{I_m}{I_w} \\
I_a &= I_m + I_w
\end{align}

Country level estimates are generated and then aggregated at the regional and global levels.



