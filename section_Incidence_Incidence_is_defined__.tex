\section{Incidence}

Incidence is defined as the number of new and recurrent (relapse) episodes of TB (all forms) occurring in a given year. Recurrent episodes are defined as a new episode of TB in people who have had TB in the past and for whom there was bacteriological confirmation of cure and/or documentation that treatment was completed. In the remainder of this technical appendix, relapse cases are referred to as recurrent cases because the term is more useful when explaining the estimation of TB incidence. Recurrent cases may be true relapses or a new episode of TB caused by reinfection. In current case definitions, both relapse cases and patients who require a change in treatment are called ‘retreatment cases’. However, people with a continuing episode of TB that requires a treatment change are prevalent cases, not incident cases. 

The case notification rate refers to new and recurrent episodes of TB notified to WHO for a given year, expressed per 100 000 population. The case notification rate for new and recurrent TB is important in the estimation of TB incidence. In some countries, however, information on treatment history may be missing for some cases. When data on treatment history are not available, recurrent cases cannot be distinguished from cases whose treatment was changed, since both are registered and reported in the category ‘retreatment’. Patients reported in the ‘unknown history’ category are considered incident TB episodes (new or relapse). 

TB incidence has never been measured at national level because this would require long-term studies among large cohorts of people (hundreds of thousands), involving high costs and challenging logistics. Notifications of TB cases provide a good proxy indication of TB incidence in countries that have both high-performance surveillance systems (for example, there is little underreporting of diagnosed cases) and where the quality of and access to health care means that few cases are not diagnosed. In the large number of countries where these criteria are not yet met, better estimates of TB incidence can be obtained from an inventory study\cite{WHO2012} (an inventory study is a survey to quantify the level of underreporting of detected TB cases; if certain conditions are met, capture-recapture methods can also be used to estimate TB incidence). To date, such studies have been undertaken in only a few countries: examples include Egypt, Iraq, Pakistan and Yemen. 

The ultimate goal is to directly measure TB incidence from TB notifications in all countries. This requires a combination of strengthened surveillance, better quantification of underreporting (i.e. the number of cases that are missed by surveillance systems) and universal access to health care. A TB surveillance checklist developed by the WHO Global Task Force on TB Impact Measurement defines the standards that need to be met for notification data to provide a direct measure of TB incidence. By August 2015, a total of 38 countries including 16 HBCs had completed the checklist.

 Methods currently used by WHO to estimate TB incidence can be grouped into four major categories (Figure 2.2). These are: 






  
  
  
  
  
  
  
  
  
  