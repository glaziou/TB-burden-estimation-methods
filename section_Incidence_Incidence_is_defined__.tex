\section{Incidence}

Incidence is defined as the number of new and recurrent (relapse) episodes of TB (all forms) occurring in a given year. Recurrent episodes are defined as a new episode of TB in people who have had TB in the past and for whom there was bacteriological confirmation of cure and/or documentation that treatment was completed. In the remainder of this technical appendix, relapse cases are referred to as recurrent cases because the term is more useful when explaining the estimation of TB incidence. Recurrent cases may be true relapses or a new episode of TB caused by reinfection. In current case definitions, both relapse cases and patients who require a change in treatment are called ‘retreatment cases’. However, people with a continuing episode of TB that requires a treatment change are prevalent cases, not incident cases. 

The case notification rate refers to new and recurrent episodes of TB notified to WHO for a given year, expressed per 100 000 population. The case notification rate for new and recurrent TB is important in the estimation of TB incidence. In some countries, however, information on treatment history may be missing for some cases. When data on treatment history are not available, recurrent cases cannot be distinguished from cases whose treatment was changed, since both are registered and reported in the category ‘retreatment’. Patients reported in the ‘unknown history’ category are considered incident TB episodes (new or relapse). 

TB incidence has never been measured at national level because this would require long-term studies among large cohorts of people (hundreds of thousands), involving high costs and challenging logistics. Notifications of TB cases provide a good proxy indication of TB incidence in countries that have both high-performance surveillance systems (for example, there is little underreporting of diagnosed cases) and where the quality of and access to health care means that few cases are not diagnosed. In the large number of countries where these criteria are not yet met, better estimates of TB incidence can be obtained from an inventory study\cite{WHO2012} (an inventory study is a survey to quantify the level of underreporting of detected TB cases; if certain conditions are met, capture-recapture methods can also be used to estimate TB incidence). To date, such studies have been undertaken in only a few countries: examples include Egypt, Iraq, Pakistan and Yemen. 

The ultimate goal is to directly measure TB incidence from TB notifications in all countries. This requires a combination of strengthened surveillance, better quantification of underreporting (i.e. the number of cases that are missed by surveillance systems) and universal access to health care. A TB surveillance checklist developed by the WHO Global Task Force on TB Impact Measurement defines the standards that need to be met for notification data to provide a direct measure of TB incidence. By August 2015, a total of 38 countries including 16 HBCs had completed the checklist.

 Methods currently used by WHO to estimate TB incidence can be grouped into four major categories (Figure 2.2). These are: 

\begin{enumerate}
\item {Case notification data combined with expert opinion about case detection gaps.} Expert opinion, elicited in regional workshops or country missions, is used to estimate levels of under-reporting and under-diagnosis. Trends are estimated using either mortality data, surveys of the annual risk of infection or exponential interpolation using estimates of case detection gaps for three years. In this report, this method is used for 120 countries that accounted for 51\% of the estimated global number of incident cases in 2014. The estimation of case detection gaps is essentially based on an in-depth analysis of surveillance data, experts provide their educated best guess about the range of plausible detection gap $g$:

$I=\frac{f(N)}{1-g}, g\in[0,1[$

where $f$ denotes a cubic spline function in countries with large year-to-year fluctuations in $N$, or else, the identity function. The incidence series are completed using assumptions about improvements in CFR over time in countries with evidence of improvements in TB control performance such as an increased detection coverage over time or improved treatment outcomes. Calibration of estimates was done using mortality data where available. 

$0 < \lvert \frac{dI}{dt} \rvert < \lvert \frac{dM}{dt} \rvert $

A full description of the methods used in regional workshops where expert opinion was systematically elicited following an in-depth analysis of surveillance data is publicly available in a report of the workshop held for countries in the African Region (in Harare, Zimbabwe, December 2010\cite{WHO}). In some countries case reporting coverage changed significantly during the period 1990-2013 as a result of surveillance reforms (e.g. disease surveillance was thoroughly reformed after the SARS epidemic in China, the MoJ sector notified cases in Russia starting in the early 2000s). Trends in incidence were derived from repeat tuberculin survey results in Bhutan, India and Yemen and for 40 countries (including countries in Eastern Europe) from trends in mortality. 

Distributions of the proportion of cases that were not reported in the three reference years were assumed to follow a Beta distribution, with the expected value $E$ and variance $V$ obtained using the method of moments\cite{Renyi2007}. Time series for the period 1990–2014 were built according to the characteristics of the levels of underreporting and under-diagnosis that were estimated for the three reference years. A cubic spline extrapolation of V and E, with knots set at the reference years, was used for countries with low-level or concentrated HIV epidemics. In countries with a generalized HIV epidemic, the trajectory of incidence from 1990 to the first reference year (usually 1997) was based on the annual rate of change in HIV prevalence. 

If there were insufficient data to determine the factors leading to time-changes in case notifications, incidence was assumed to follow a horizontal trend going through the most recent estimate of incidence. 

Limitations of the method included a generally small number of interviewed experts; lack of clarity about vested interests when eliciting expert opinion; lack of recognition of over-reporting (due to over-diagnosis,  e.g. in some countries of the former Soviet Union using large-scale systematic population screening); 
incomplete data on laboratory quality and high proportion of patients with no bacteriological confirmation of diagnosis are a potential source of error in estimates.


\item {Results from national TB prevalence surveys.} Incidence is estimated using prevalence survey results combined with either a dynamic model or estimates of the duration of disease. This method is used for 19 countries that accounted for 46\% of the estimated global number of incident cases in 2014.

Incidence can be estimated using measurements from national surveys of the prevalence of TB disease combined with estimates of the duration of disease. Incidence is estimated as the prevalence of TB divided by the average duration of disease. 

In practice, the duration of disease cannot be directly measured. For example, measurements of the duration of symptoms in prevalent TB cases that are detected during a prevalence survey are systematically biased towards lower values, since active case-finding truncates the natural history of undiagnosed disease. Measurements of the duration of disease in notified cases ignore the duration of disease among non-notified and untreated cases. 

Literature reviews commissioned by the WHO Global Task Force on TB Impact Measurement have provided estimates of the duration of disease in untreated TB cases from the pre-chemotherapy era (before the 1950s). The best estimate of the mean duration of disease (for smear-positive cases and smear-negative cases combined) in HIV-negative individuals is about three years. However, the proportion of incident cases that remain untreated is unknown. There are few data on the duration of disease in HIV-positive individuals. 

The assumed distributions of disease durations are shown in Table.

\begin{table} 
    \begin{tabular}{ c c }
        \hline
        Case category & Distribution of disease duration (year) \\ 
        \hline
        Notified, HIV-negative & Uniform $(0.2 - 2)$ \\ 
        Not notified, HIV-negative & Uniform $(1 - 4)$ \\ 
        Notified, HIV-positive & Uniform $(0.01 - 1)$ \\ 
        Not notified, HIV-positive & Uniform $(0.01 - 0.2)$ \\ 
        \hline
    \end{tabular} 
    \caption{Distribution of disease duration by case category} 
\end{table}

In countries with a high burden of HIV ...


A second approach consists in estimating disease duration through a simple dynamical model with three compartments: susceptibles ($S$), untreated TB ($U$) and treated TB ($T$). The size of $U$ and $T$ is estimated from the prevalence survey findings. Transitions from $U$ to $T$ are determined as follows:

$\frac{dU}{dt} = I - (\mu_u + \theta_u + \delta)U$

Where $I$ denotes Incidence, μu+θu denote mortality (untreated) and self-cure (untreated), respectively, $\delta$ denotes the rate of removal from U through detection and treatment, $\mu_T + \theta_T$ denote mortality (treated) and cure, respectively. At equilibrium, the above two equations simplify to: 

$I = frac{U}{d_U}$

$\delta U = \frac{T}{d_T}$

And disease duration (untreated) is obtained from $d_U=(1-\pi)U_T d_T$, where $\pi$ is the proportion of incidence that dies or self-cures before treatment. $\pi$ is assumed distributed uniform with bounds 0 and 0.1. Table 6 shows estimates of incidence from four recent prevalence surveys using this method. 

\begin{table} 
    \begin{tabular}{ c c c c c c }
         & $U$ & $T$ & Prevalence (10^{-3}) & Duration(year) & Incidence(10^{-3}y^{-1}) \\ 
        Cambodia 2002 & 260 & 42 & 12 (10-15) & 2.9 (1.9-4) & 4 (2.5-5.8) \\ 
        Cambodia 2011 & 205 & 80 & 8.3 (7.1-9.8) & 1.2 (0.8-1.6) & 6.7 (4.5-9.3) \\ 
        Myanmar 2009 & 300 & 79 & 6.1 (5-7.5) & 1.8 (1.1-1.6) & 3.3 (2-4.8) \\ 
        Thailand 2012 & 136 & 60 & 2.5 (1.9-3.5) & 1.1 (0.5-1.6) & 2.3 (1-3.5) \\ 
    \end{tabular} 
    \caption{Incidence estimation based on $U/T$ ratio
} 
\end{table}

Estimates suffer from considerable uncertainty, mostly because surveys are not powered to estimate the number of prevalent TB cases on treatment with great precision. Further, in most surveys, cases found on treatment during the survey do not have a bacteriological status at onset of treatment documented based on the same criteria as survey cases (particularly when culture is not performed routinely). This method does not provide unbiased estimates of incidence.

The above two methods to derive incidence from prevalence are compared below.

\begin{table} 
    \begin{tabular}{ c c c c }
         & Prevalence(10^{-3}) & Incidence - method 1(10^{-3} y^{-1}) & Incidence - method 2(10^{-3} y^{-1}) \\ 
        Cambodia 2002 & 12 (10-15) & 4 (2.5-5.8) & 2.2 (1.5 – 2.9) \\ 
        Cambodia 2011 & 8.3 (7.1-9.8) & 6.7 (4.5-9.3) & 3.8 (2.2 – 5.8) \\ 
        Myanmar 2009 & 6.1 (5-7.5) & 3.3 (2-4.8) & 3.4 (2 – 5.1) \\ 
        Thailand 2012 & 2.5 (1.9-3.5) & 2.3 (1-3.5) & 1.1 (0.7 – 1.6) \\ 
    \end{tabular} 
    \caption{Estimates of incidence derived from prevalence survey results, based on two estimation methods.} 
\end{table}

It is not clear which method will perform better as validation would require a measurement of incidence. The second method requires a sufficient number of cases on treatment at the time of the survey (at least 30 cases) to be applied. When the number of cases on treatment is too small, the amount of propagated uncertainty renders estimates of incidence unusable. 




\item {Notifications in high-income countries adjusted by a standard factor to account for under-reporting and under-diagnosis.} This method is used for 73 countries (all high-income countries except the Netherlands and the United Kingdom), which accounted for 3\% of the estimated global number of incident cases in 2014.

In the absence of country-specific data on the quality and coverage of TB surveillance systems, it was assumed that TB surveillance systems from countries in the high-income group performed similarly well, although the model does allow for stochastic fluctuations. The exceptions were the United Kingdom and the Netherlands, where the underreporting of TB cases has been measured using inventory studies and capture–recapture modelling\cite{Anderson2010}\cite{17156496}. For these two countries, the results from these studies were used to measure TB incidence directly.


\item {Results from inventory/capture-recapture studies.} This method is used for 5 countries: Egypt, Iraq, the Netherlands, the United Kingdom and Yemen. They accounted for 0.5\% of the estimated global number of incident cases in 2014. 
\end{enumerate}




  
  
  
  
  
  
  
  
  
  